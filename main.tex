% 中文请用ctexbeamer, 个人觉得行距linespread宽一点看起来比较舒服。
\documentclass[aspectratio=1610,linespread=1.4,t]{ctexbeamer}

% If this is an English-only deck, you don't need ctexbeamer nor the wider linespread
% \documentclass[aspectratio=1610,t]{beamer}
% But load fontspec so that you can load OTF/TTF fonts if you want to!
% \usepackage{fontspec} 

% beamerthemeMirage.sty v1.0 2024/04/14
% by LianTze Lim (liantze@gmail.com)
% A beamer theme inspired by Zhou Shen's `Mirage' song  poster
% 基于周深《反深代词》先行曲《蜃楼》歌曲海报色调的beamer主题
\usetheme{Mirage}           % Default = dark mode
% \usetheme[light]{Mirage}  % Alternative light mode 

% Mirage theme loads the fontawesome5 package

\title{Mirage Theme 蜃楼}
\subtitle{《反深代词》已全平台上线}
\author{董敏儿}
\institute{(懂自懂)}
\date{2024/05/19}

% These fonts will work with pdflatex, xelatex, lualatex
\usepackage[lining,tabular]{carlito}
\usepackage{caladea}

% But unicode and TTF/OTF fonts will only work with xelatex, lualatex
\usepackage{unicode-math}
% \setmathfont{STIX Two Math}
% \setmathfont{Erewhon Math}
\setmathfont{Fira Math}

\begin{document}

\frame{\maketitle}

\section{Intro}
\begin{frame}
\frametitle{Introduction}

\begin{itemize}
    \item 看机械的白鸽\faDove{} 从空中飞过 
    \item 要如何点睛\faEye[regular] 它才堪称鲜活

\begin{enumerate}
    \item 数字的晨昏\faCloudSun{} 是否更缤纷\faCloudMoon
    \item 仿生的情人\faGrinHearts{} 是否更忠贞\faGrin*[regular]
\end{enumerate}

    \item 推开一扇门\faDoorOpen{} 还有万千重门\faDoorClosed{\small\faDoorClosed}{\footnotesize\faDoorClosed}{\scriptsize\faDoorClosed}{\tiny\faDoorClosed}
\end{itemize}

\end{frame}

\subsection{Lists}
\begin{frame}
\frametitle{那来自过去 古老的眼神}
    \begin{enumerate}
        \item 如何能辨认 此刻是幻是真
        \begin{enumerate}
	        \item 人造的天分 是否算慧根
	        \begin{enumerate}
		        \item 克隆的肉身 是否有灵魂
		        \item 永远在追问 却从来都没结论
	        \end{enumerate}
        \end{enumerate}
        \item \alert{Can it be real}
    \end{enumerate}

\end{frame}

\section{Blocks}

\begin{frame}[c]
\frametitle{智慧的天梯从来都没尽头}
\begin{columns}
\begin{column}{.4\textwidth}
\begin{pullquote}
    Can it be real\\
    The world is a mirage
\end{pullquote}
\end{column}

\begin{column}{.53\textwidth}
\setbeamercolor{pullquote}{fg=MirageBlue}
\renewcommand{\MiragePullquoteOpen}{\hskip-.2\ccwd『}
\begin{pullquote}
在电幻的荒丘 寻真实的绿洲\\
渺小得如蜉蝣 也仰望着宇宙
\end{pullquote}
\end{column}
\end{columns}
\end{frame}


% \renewcommand{\MirageFrametitlePrefix}{\faDove}
% \renewcommand{\MirageProofPrefix}{\faCheckSquare[regular]}
% \renewcommand{\MirageTheoremPrefix}{\faCogs}

\begin{frame}[allowframebreaks]{各种block 123}
    \begin{exampleblock}{算了我也不知道在写什么,do you?}
    Now solve $x = \frac{-b \pm \sqrt{b^2 -4ac}}{2a}$. 对各位同学来说应该挑战不大。
    \end{exampleblock}
    
    \begin{alertblock}{算了我也不知道在写什么,do you?}
    \[ x = \frac{-b \pm \sqrt{b^2 -4ac}}{2a}, \quad\therefore \alpha \neq \Omega \]
    \end{alertblock}
    
    \begin{block}{算了我也不知道在写什么,do you?}
    \[ x = \frac{-b \pm \sqrt{b^2 -4ac}}{2a}, \quad\therefore \alpha \neq \Omega \]
    \end{block}
    
    \begin{proof}
    显而易见,$1+1=2$.
    \end{proof}

    \begin{theorem}
    有一件很美好的事情将要发生,它终会发生。
    \end{theorem}
    
    \begin{definition}
    有一件很美好的事情将要发生,它终会发生。
    \end{definition}
\end{frame}


\end{document}
